% settings.tex
%*********************************************************************************

%*********************************************************************************
% Comandi personali
%*********************************************************************************
\newcommand{\myname}{Leonardo~Di~Donato}
\newcommand{\mytitle}{Pure awesomeness}
\newcommand{\mydegree}{M.sc thesis}
\newcommand{\myuni}{University of Github}
\newcommand{\myfaculty}{Internet}
\newcommand{\mydepartment}{Coders}
\newcommand{\myprof}{Prof.~N.~Surname}
\newcommand{\myocorrelator}{Dott.~N~Surname}
\newcommand{\mylocation}{Somewhere}
\newcommand{\mytime}{September 2013}
\newcommand{\mysubject}{Awesomeness}
\newcommand{\mykeywords}{thesis, tex, latex, pdflatex, templating}

%*********************************************************************************
% TOC settings
%*********************************************************************************
% \setcounter{secnumdepth}{5}
% \setcounter{tocdepth}{5}

%*********************************************************************************
% Chapters style
%*********************************************************************************
% Chapter number style: decomment if you want it with the same corpus font
% \renewcommand{\chapterNumber}{%
% \fontsize{70}{70}\usefont{\encodingdefault}{\sfdefault}{b}{n}%
% }%

% Workaround: multi-line titles
\renewcommand\formatchapter[1]{%
\begin{minipage}[b]{0.15\linewidth}
\chapterNumber
\end{minipage}%
\begin{minipage}[b]{0.70\linewidth}%length of the second row
\raggedright\spacedallcaps{#1}
\end{minipage}
}

%*********************************************************************************
% Analytical index
%*********************************************************************************
\makeindex

% Balance columns of the last page
% \let\orgtheindex\theindex
% \let\orgendtheindex\endtheindex
% \def\theindex{%
% \def\twocolumn{\begin{multicols}{2}}%
% \def\onecolumn{}%
% \clearpage
% \orgtheindex
% }
% \def\endtheindex{%
% \end{multicols}%
% \orgendtheindex
% }

% Indexing commands
\newcommand*{\keyword}[2][]{#2\index{#2@#2#1}}% as simple keyword
\newcommand{\keywordsub}[2][]{#2\index{#1!#2}}% as categorized keyword
\newcommand{\keywordpagb}[1]{\keyword[|textbf]{#1}}% as simple keyword with bold page number
\newcommand{\keywordpagi}[1]{\keyword[|textit]{#1}}% as simple keyword with italic page number
\newcommand{\keywordbold}[1]{#1\index{#1@\textbf{#1}}}% as bold keyword
\newcommand{\keyworditalic}[1]{#1\index{#1@\textsl{#1}}}% as italic keyword

%*********************************************************************************
% Impostazioni di amsmath, amssymb, amsthm
%*********************************************************************************
% Force the equation numbers to be always the same size
\makeatletter
\renewcommand{\maketag@@@}[1]{\hbox{\m@th\normalsize\normalfont#1}}%
\makeatother

% operators
\DeclareMathOperator*{\argmax}{arg\,max}
\DeclareMathOperator*{\argmin}{arg\,min}

% comandi per gli insiemi numerici (serve il pacchetto amssymb)
\newcommand{\numberset}{\mathbb}
\newcommand{\N}{\numberset{N}}
\newcommand{\R}{\numberset{R}}

% comandi per gli insiemi
\newcommand{\set}[1]{\mathbf{#1}}
\newcommand{\setel}[1]{\mathnormal{#1}}  % o \mathrm o \mathit

% comandi per i vettori
\newcommand{\vect}[1]{\set{#1}}
\newcommand{\vectel}[1]{\mathrm{#1}}

% comandi per i simboli matematici che denotano schemi, categorie, concetti
\newcommand{\conceptsym}[1]{\mathcal{#1}}

% un ambiente per i sistemi
\newenvironment{sistema}%
  {\left\lbrace\begin{array}{@{}l@{}}}%
  {\end{array}\right.}

% comandi per le lettere greche
% ...

% reference for equation terms. use \underbrace{equation}_{\myterm{termA}}
\newcounter{term}
\renewcommand*{\theterm}{(\alph{term})}
\AtBeginDocument{%
  \let\mylabel\label
}
\newcommand{\myterm}[1]{%
  \begingroup % keep the effects of \refstepcounter local
    \refstepcounter{term}%
    \mylabel{#1}%
    \text{\theterm}%
  \endgroup
}

%*********************************************************************************
% Nuovi ambienti: definizioni, teoremi etc. etc.
%*********************************************************************************
% definizioni (serve il pacchetto amsthm)
\newtheoremstyle{classicdef}% Nome
{12pt}% Spazio che precede l�enunciato
{12pt}% Spazio che segue l�enunciato
{}% Stile del font dell�enunciato
{}% Rientro (se vuoto, non c�� rientro,
% \parindent = rientro dei capoversi)
{\scshape}% Stile del font dell�intestazione
{:}% Punteggiatura che segue l�intestazione
{.5em}% Spazio che segue l�intestazione:
% " " = normale spazio inter-parola;
% \newline = a capo
{}% Specifica l�intestazione dell�enunciato
% (normalmente viene lasciata vuota)

\theoremstyle{definition}
\newtheorem{definizione}{Definizione}
% \newtheorem{osservazione}[definizione]{Osservazione}
% \newtheorem{definizione}{Definizione}[section]
\newtheorem{osservazione}{Osservazione}[definizione]

% esempi
\theoremstyle{definition}
\newtheorem{esempio}{Esempio}[definizione]

% note about definitions
\theoremstyle{remark}
\newtheorem{notabene}{Nota}[definizione]

% note of section
\theoremstyle{remark}
\newtheorem{nota}{Nota}[section]

% teoremi (serve il pacchetto amsthm)
\newtheoremstyle{classicthm}% Nome
{12pt}% Spazio che precede l�enunciato
{12pt}% Spazio che segue l�enunciato
{\itshape}% Stile del font dell�enunciato
{}% Rientro (se vuoto, non c�� rientro,
% \parindent = rientro dei capoversi)
{\scshape}% Stile del font dell�intestazione
{:}% Punteggiatura che segue l�intestazione
{.5em}% Spazio che segue l�intestazione:
% " " = normale spazio inter-parola;
% \newline = a capo
{}% Specifica l�intestazione dell�enunciato
% (normalmente viene lasciata vuota)

\theoremstyle{plain}
\newtheorem{teorema}{Teorema}[chapter]
\newtheorem*{teorema*}{Teorema}
\newtheorem{cor}[teorema]{Corollario}
\newtheorem{lem}[teorema]{Lemma}
\newtheorem{prop}[teorema]{Proposizione}
\newtheorem{oss}[teorema]{Osservazione}

% leggi (serve il pacchetto amsthm)
\newtheorem{legge}{Legge}
\newtheorem{decreto}[legge]{Decreto}

%*********************************************************************************
% Impostazioni di acronym
%*********************************************************************************
\newcommand{\acroname}{Acronimi}
% \renewcommand*{\acsfont}[1]{\textssc{#1}}                 % for MinionPro
\renewcommand*{\acsfont}[1]{\textsmaller{#1}}               % customize font for long version acronyms [works only if footnote not activate]
\renewcommand*{\acffont}[1]{#1}                             % idem, but for short version of acronyms
\renewcommand{\bflabel}[1]{{#1}\hfill}                      % fix the list of acronyms
\makeatletter                                               % macro that tweeks acronym package to rendere lowercase or not
\newif\if@in@acrolist
\AtBeginEnvironment{acronym}{\@in@acrolisttrue}
\newrobustcmd{\ul}[2]{\if@in@acrolist#1\else#2\fi}          % \ul{C}{c}iao defines an uppercase and lowercase variant of the same acronym
\newcommand{\ACF}[1]{{\@in@acrolisttrue\acf{#1}}}           % \ACF{<acronym>} force the defined capitalized variants of acronym letters/words
\newcommand{\ACL}[1]{{\@in@acrolisttrue\acl{#1}}}           % \ACL{<acronym}
\makeatother

%*********************************************************************************
% Impostazioni di biblatex
%*********************************************************************************
\defbibheading{bibliography}{%
\cleardoublepage
\manualmark
\phantomsection
\addcontentsline{toc}{chapter}{\tocEntry{\bibname}}
\chapter*{\bibname\markboth{\spacedlowsmallcaps{\bibname}}
{\spacedlowsmallcaps{\bibname}}}}

%*********************************************************************************
% Definizione di un linguaggio di pseudo-codice
%*********************************************************************************
\lstdefinelanguage{pseudo}{
    morekeywords=%
    {
        for,foreach,return,if,else,ifelse,max,min,in,len,log,index,var,function,seqlen,as,new,while,unique,which
    },
    morecomment=[l][\color{Green}\ttfamily]{//},
    morecomment=[s][\color{Green}\ttfamily]{/*}{*/},
    morestring=[b][\color{Red}\rmfamily]",
    moredelim=[is][\color{Orange}]{|*}{*|}
}

%*********************************************************************************
% Impostazioni di listings
%*********************************************************************************
\lstset{
    %language=pseudo,%[LaTeX]Tex,%C++,
    basicstyle=\normalsize\ttfamily,
    % identifierstyle=\color{NavyBlue},
    % keywordstyle=\bfseries,%\color{RoyalBlue},%\bfseries,
    % commentstyle=\color{Green}\ttfamily,
    % stringstyle=\rmfamily,
    numbers=left,%none
    numberstyle=\scriptsize,%\tiny
    stepnumber=1,
    numbersep=8pt,
    showstringspaces=false,
    breaklines=true,
    frameround=ftff,%fttt,%
    frame=TB,%lines,
    tabsize=1,
    escapeinside={<ls>}{<le>}
}

%*********************************************************************************
% Impostazioni di hyperref
%*********************************************************************************
\hypersetup{%
    % hyperfootnotes=false,pdfpagelabels,
    %draft,	% = elimina tutti i link (utile per stampe in bianco e nero)
    colorlinks=true, linktocpage=true, pdfstartpage=1, pdfstartview=FitV,%
    % decommenta la riga seguente per avere link in nero (per esempio per la stampa in bianco e nero)
    % colorlinks=false, linktocpage=false, pdfborder={0 0 0}, pdfstartpage=1, pdfstartview=FitV,%
    breaklinks=true, pdfpagemode=UseNone, pageanchor=true, pdfpagemode=UseOutlines,%
    plainpages=false, bookmarksnumbered, bookmarksopen=true, bookmarksopenlevel=1,%
    hypertexnames=true, pdfhighlight=/O,%nesting=true,%frenchlinks,%
    urlcolor=webbrown, linkcolor=RoyalBlue, citecolor=webgreen, %pagecolor=RoyalBlue,%
    urlcolor=Black, linkcolor=Black, citecolor=Black, %pagecolor=Black,%
    pdftitle={\mytitle},%
    pdfauthor={\textcopyright\ \myname, \myuni, \myfaculty},%
    pdfsubject={\mysubject},%
    pdfkeywords={\mykeywords},%
    pdfcreator={pdfLaTeX},%
    pdfproducer={LaTeX with hyperref, classicthesis and arsclassica}%
}

%*********************************************************************************
% Impostazioni di graphicx
%*********************************************************************************
\graphicspath{{images/}} % cartella dove sono riposte le immagini

%*********************************************************************************
% A4 optimized margins
%*********************************************************************************
\areaset[current]{336pt}{750pt}
\setlength{\marginparwidth}{7em}
\setlength{\marginparsep}{2em}%

%*********************************************************************************
% Utilities
%*********************************************************************************
% make first letter uppercase
\makeatletter
\def\upcase{\expandafter\makeupcase}
\def\makeupcase#1{\uppercase{#1}}
\makeatother

% make first letter lowercase
\makeatletter
\def\lwcase{\expandafter\makelwcase}
\def\makelwcase#1{\lowercase{#1}}
\makeatother

%*********************************************************************************
% References
%*********************************************************************************
\addto\extrasitalian{%
  \def\equationautorefname{equazione}%
}
\addto\extrasitalian{%
  \def\chapterautorefname{capitolo}%
}
\addto\extrasitalian{%
  \def\sectionautorefname{sezione}%
}
\addto\extrasitalian{%
  \def\subsectionautorefname{sottosezione}%
}
\addto\extrasitalian{%
  \def\subsubsectionautorefname{sottosezione}%
}
\addto\captionsitalian{%
    \renewcommand{\lstlistingname}{Algoritmo}%
}
\addto\captionsitalian{%
    \renewcommand{\lstlistlistingname}{Elenco degli algoritmi}%
}
\providecommand*{\lstlistingautorefname}{algoritmo}
\providecommand*{\lstnumberautorefname}{linea}

%*********************************************************************************
% Custom references for \autoref
%*********************************************************************************
\newcommand{\definizioneautorefname}{definizione}
\newcommand{\osservazioneautorefname}{osservazione}
\newcommand{\notabeneautorefname}{nota}
\newcommand{\esempioautorefname}{esempio}
\newcommand{\termautorefname}{termine}
\newcommand{\teoremaautorefname}{teorema}

%*********************************************************************************
% Set footnote's continuos numbering and linking capabilities (using cleveref and chngcntr packages)
%*********************************************************************************
\counterwithout{footnote}{chapter}
\crefformat{footnote}{#2\footnotemark[#1]#3}

%*********************************************************************************
% Define your commands here
%*********************************************************************************
% ...
\newcommand{\omissis}{\dots\negthinspace}
\newcommand{\doublequotes}[1]{``#1''}
\newcommand{\caporali}[1]{�#1�}
\newcommand{\ie}{i.\,e.}
\newcommand{\Ie}{I.\,e.}
\newcommand{\eg}{e.\,g.}
\newcommand{\Eg}{E.\,g.}

%*********************************************************************************
% Hypenation exceptions
%*********************************************************************************
\hyphenation{Fortran ma-cro-istru-zio-ne nitro-idrossil-amminico}
