% !TEX encoding = UTF-8
% !TEX TS-program = pdflatex
% !TEX root = ../arsclassica.tex
% !TEX spellcheck = it-IT

%*******************************************************
% Introduction
%*******************************************************
\cleardoublepage
\pdfbookmark{Introduction}{Introduction}
\chapter*{Introduction}\label{cap:intro}

\lipsum[1]

Lorem ipsum dolor sit amet, consectetuer adipiscing elit.
% \begin{description}
% \item[{\hyperref[cap:lorem]{Il primo capitolo}}]
% offre una visione d'insieme della storia di \LaTeX{} e ne vengono presentate le idee di fondo.
% \item[{\hyperref[cap:ipsum]{Il secondo capitolo}}]
% spiega le operazioni, veramente semplici, per installare \LaTeX{} sul proprio calcolatore.
% \item[{\hyperref[cap:dolor]{L'appendice A}}] descrive  sinteticamente le principali norme tipografiche della lingua italiana, utili nella composizione di articoli, tesi o libri.
% \end{description}

\begin{description}
\item[{\hyperref[cap:one]{First chapter}}]
talks about \omissis{}.
\item[{\hyperref[cap:two]{Second chapter}}]
talks about \omissis{}.
\item[{\hyperref[cap:three]{Third chapter}}]
talks about \omissis{}.
\end{description}

% \lipsum[2]
